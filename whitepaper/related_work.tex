\section{Related Work}

This section presents the projects that inspired the App Coins protocol, either because of the technology employed or by presenting concepts that empower the protocol. We first give a brief overview of each project and after that we explain how each of our uses cases benefit from the contributions of each of these projects.

\subsection{Related Projects}
\subsubsection{Basic Attention Token}

The BAT project aims to revolutionise the digital advertising landscape by proposing a "decentralised, transparent digital ad exchange based on Blockchain" \textbf{[REF BAT PAPER]}. Their proposal is constituted by two components:
\begin{itemize}
	\item Brave: a browser that blocks third-party ads and trackers, which decreases webpages load time and ensures anonymity, while also building a ledger system that tracks users attention to ads in order to correctly reward publishers and advertisers
	\item BAT: a token for the decentralised ad exchange connecting advertisers, publishers and users, while rewarding users for their attention.
\end{itemize}

In their proposal, BAT wants to eliminate the middlemen between advertisers and publishers, pay for user attention instead of CPM/clicks, provide faster webpage loads and ads tuned to user preferences, amongst other features. \\

By taking out the middlemen, BAT avoids draining of resources by agencies, DSPs, exchanges, ad networks, and others, while also eliminating part of the complexity of having to deal with this huge ecosystem that is in place today. The gain in resources by eliminating the resource-draining players enables the sharing of resources by the fundamental, value-generating  players in the flow: advertisers, publishers and users. Since there are less resources being wasted in middlemen, there is more available to be employed in processes that increase the value to the end user. BAT also proposes to use machine learning at the browser level to serve tailored ads to users, instead of serving ads with questionable value. In addition, users are rewarded by their attention, which the project states as the "rare quantity", since the information available is far greater that the available attention each user has to give. \\

%XXX the last sentence is very difficult to digest. I don't get the meaning...

Today, publishers are paid based on clicks on ads. BAT proposes to start rewarding publishers based on the attention users give to ads, by keeping track of the user attention on a ledger system implemented in Brave, while always maintaining users' anonymity. User attention - a very valuable asset - is not being rewarded correctly and users do not get anything while they navigate webpages and see the ads. The solution proposes that users also start receiving rewards for time spent seeing the ads while navigating, based on the amount of time they spend looking at them. \\

In order to reduce fraud, they propose approaches - calling them Basic Attention Metrics (BAM) - to correctly identify users paying attention to ads. When the user attention is identified, it is saved in an anonymous way. At the same time BAT also ensures that users do not get rewarded by paying attention to the same ad more than once. They define a \textit{proof-of-attention} algorithm, which ensures that a user can only see and get attributed to an ad once and maintains users' anonymity. This is achieved by using \textsf{ANONIZE} \textbf{[REF ANONYZE PAPER]} algorithm in a first stage. According to the authors, the algorithm is "the first implementation of a provably-secure multiparty protocol that scales to handle millions of users".  The BAT team says that they may also invest into using algorithms such as BOLT \textbf{[REF BOLT PAPER]}, zkSNARKs \textbf{[REF SNARKs PAPERS]} and STARKs \textbf{[REF STARKs PAPER]} to protect users' privacy.

%XXX I say proof-of-attention algorithm, maybe there is a better word than algorithm. metrics?

\subsubsection{Kin}

The Kin project intends to create the "first open and sustainable alternative ecosystem of digital services
for our daily lives" \textbf{[REF KIN PAPER]}. In order to achieve this, a new cryptocurrency Kin is created to be used within this ecosystem of digital services. \\

Since Kin is being developed by Kik, a popular chat app with already millions of users, Kik will integrate Kin to showcase the possibilities of having an ecosystem of connected digital services. New partners joining the ecosystem will create a network effected, boosting the value of Kin. Kik will develop two main components of the new ecosystem:
\begin{itemize}
	\item Kin Reward Engine
	\item Kin Foundation
\end{itemize}

The Kin Reward Engine is going to create incentives for other digital services to adopt Kin. The majority of the Kin supply will be allocated to Kin Reward Engine and periodically will unlock and distribute a certain amount of Kin amongst the digital services within the ecosystem. The amount each digital service receives depends on the amount of Kin used by them. \\

Kin Foundation is the entity that will oversee the growth of the ecosystem, as well as administer the Kin Reward Engine. In time, the Kin Foundation will transition the entire ecosystem, including the Kin Reward Engine to a fully decentralised and autonomous network. When this happens, Kin Foundation main responsabilities will be helping onboarding new partners and overseeing development of fundamental components such as identity and reputation management, cryptocurrency wallets, and compliance solutions. \\

In the end, Kin wants to develop an ecosystem that is open and fair, where users benefit from a vast and diverse digital experience, being able to transition between services with almost no effort. Providers will be able to compete for compensation within the ecosystem.

\subsubsection{Monetha}

The Monetha project aims to change how e-commerce is done and how merchants can reach customer by "creating a universal decentralised trust and reputation solution working flawlessly together with mobile payments processing on the Ethereum blockchain leveraging smart contract technology" \textbf{[REF MONETHA PAPER]}. \\

Currently, merchants wanting to sell online face two options: creating their own website or joining the big marketplaces, such as Amazon, Ebay, Alibaba, etc. The former requires a very significant investment in brand creation and advertising in order to have customers going directly to their site. This is due to the fact that customers tend to buy on places they trust and that have good reputation. If there is a merchant with a website no-one has heard about, there will be a trust barrier for the customers. On the other hand, joining big marketplaces has the advantage of not needing much advertising and branding but there is still the need to create reputation and trust amongst customers. Merchants accomplish this by providing high quality products delivered within the agreed time and, in turn, receive positive reviews. The problem with this approach is that the reputation a merchant is able to build on one marketplace is not transferable to others. If a merchant wants to sell on several marketplaces, since they are all disconnected, the merchant needs to build a reputation amongst customers on all of them.

In addition to the reputation problem, big marketplaces also impose very complex transaction processes. The transactions settlements are troublesome to the customer due to the high number of steps required, which Monetha states that can go up to 16. An additional problem is the high transaction fees, both regarding the number of fees, with Monetha claiming that can go up to 15, and the amount needed to be paid by the merchant. High fees can deter a merchant with small margins to sell products in the marketplaces, leaving the merchant with the option to create a brand and website, which is also an expensive option as explained above. Not only merchants need to pay high fees but there is also long transaction times between the marketplace and the merchant. Since there are so many parties involved in a transaction, the settlement can take up to 3 days, or a week for international payments. Moreover, marketplaces often hold payments for a week because of the high probability of chargebacks.

Monetha proposes a solution composed by a decentralised trust and reputation system together with payments powered by blockchain technology using Ethereum. Instead of having marketplaces with their own reputation system, Monetha proposes that merchants in the network build their reputation from transactions, claims and reviews. Furthermore, the since the network is shared by all merchants, it is as if the network would be a huge marketplace where any customer can see the reputation and trust score of every merchant and vice-versa. Instead of having to build reputation in several different systems, their reputation is build automatically and available to everyone.

Since payments are done through Ethereum, settlements are much simpler and the connection is direct between the merchant and the customer. With fewer steps in the settlement comes the advantage of smaller settlement times, with the merchant getting the money within just a few minutes. Fees are also much smaller, with a fixed cost of 1.5\% of transaction fee. Finally, since there is no centralised marketplace, there is also no holding of payments.

\subsection{Projects Contributions}
\subsubsection{Advertising}

\subsubsection{IAB}

\subsubsection{Developers Reputation}






